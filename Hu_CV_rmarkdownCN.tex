\documentclass[10.5pt,]{article}

\usepackage{multicol} % for multiple columns



\usepackage{lmodern}
\usepackage{amssymb,amsmath}
\usepackage{ifxetex,ifluatex}
\usepackage{fixltx2e} % provides \textsubscript
\ifnum 0\ifxetex 1\fi\ifluatex 1\fi=0 % if pdftex
  \usepackage[T1]{fontenc}
  \usepackage[utf8]{inputenc}
\else % if luatex or xelatex
  \ifxetex
    \usepackage{xltxtra,xunicode} %originally, \usepackage{mathspec}. This change is to produce Chinese
  \else
    \usepackage{fontspec}
  \fi
  \defaultfontfeatures{Ligatures=TeX,Scale=MatchLowercase}
\fi
% use upquote if available, for straight quotes in verbatim environments
\IfFileExists{upquote.sty}{\usepackage{upquote}}{}
% use microtype if available
\IfFileExists{microtype.sty}{%
\usepackage{microtype}
\UseMicrotypeSet[protrusion]{basicmath} % disable protrusion for tt fonts
}{}
\usepackage[margin=1in]{geometry}




\setlength{\emergencystretch}{3em}  % prevent overfull lines
\providecommand{\tightlist}{%
  \setlength{\itemsep}{0pt}\setlength{\parskip}{0pt}}
\setcounter{secnumdepth}{0}
% Redefines (sub)paragraphs to behave more like sections
\ifx\paragraph\undefined\else
\let\oldparagraph\paragraph
\renewcommand{\paragraph}[1]{\oldparagraph{#1}\mbox{}}
\fi
\ifx\subparagraph\undefined\else
\let\oldsubparagraph\subparagraph
\renewcommand{\subparagraph}[1]{\oldsubparagraph{#1}\mbox{}}
\fi

% Now begins the stuff that I added.
% ----------------------------------

% Custom section fonts
\usepackage{sectsty}
\sectionfont{\rmfamily\mdseries\large\bf}
\subsectionfont{\rmfamily\mdseries\normalsize\itshape}


% Make lists without bullets
\renewenvironment{itemize}{
  \begin{list}{}{
    \setlength{\leftmargin}{1.5em}
  }
}{
  \end{list}
}


% Make parskips rather than indent with lists.
\usepackage{parskip}
\usepackage{titlesec}

	\usepackage{ctex}
	% less space for the Chinese format
	\titlespacing\section{0pt}{1pt plus 4pt minus 2pt}{1pt plus 2pt minus 2pt}
	\titlespacing\subsection{0pt}{5pt plus 4pt minus 2pt}{0pt plus 2pt minus 2pt}


% Use fontawesome. Note: you'll need TeXLive 2015. Update.
\usepackage{fontawesome}

% Fancyhdr, as I tend to do with these personal documents.
\usepackage{fancyhdr,lastpage}
\pagestyle{fancy}
\renewcommand{\headrulewidth}{0.0pt}
\renewcommand{\footrulewidth}{0.0pt}
\lhead{}
\chead{}
\rhead{}
\lfoot{
\cfoot{\scriptsize  胡悦 - CV }}
\rfoot{\scriptsize \thepage/{\hypersetup{linkcolor=black}\pageref{LastPage}}}

% Always load hyperref last.
\usepackage{hyperref}
\PassOptionsToPackage{usenames,dvipsnames}{color} % color is loaded by hyperref

\hypersetup{unicode=true,
            pdftitle={胡悦:  CV (Curriculum Vitae)},
            pdfauthor={胡悦},
            colorlinks=true,
            linkcolor=blue,
            citecolor=Blue,
            urlcolor=blue,
            breaklinks=true, bookmarks=true}
\urlstyle{same}  % don't use monospace font for urls

\begin{document}


\centerline{\huge \bf 胡悦}

\vspace{2 mm}

\hrule

\vspace{2 mm}


\moveleft.5\hoffset\centerline{313 Schaeffer Hall, 20E Washington Street, Iowa City, IA, 52242}
\moveleft.5\hoffset\centerline{ \faEnvelopeO \hspace{1 mm} \href{mailto:}{\tt \href{mailto:yue-hu-1@uiowa.edu}{\nolinkurl{yue-hu-1@uiowa.edu}}} \hspace{1 mm}  \faPhone \hspace{1 mm}  +01-803-250-1686  \hspace{1 mm}  \faGithub \hspace{1 mm} \href{http://github.com/sammo3182}{\tt sammo3182} \hspace{1 mm}    \faGlobe \hspace{1 mm} \href{http://sammo3182.github.io}{\tt sammo3182.github.io}   }

\vspace{2 mm}

\hrule


\section{教育经历}

\subsection{学历}

\begin{itemize}
\tightlist
\item
  政治学博士, 美国爱荷华大学.\hfill 2018-05 (预期)

  \begin{itemize}
  \tightlist
  \item
    \footnotesize 博士论文: ``Dominant Language: A Political
    Stabilizer''

    \begin{itemize}
    \tightlist
    \item
      导师:唐文方 (Wenfang Tang)
    \item
      博士论文委员会:William M. Reisinger, Frederick Solt, Caroline J.
      Tolbert, and David Cassels Johnson (教育系)
    \end{itemize}
  \end{itemize}
\item
  信息学辅修证书, 美国爱荷华大学\hfill 2016-12
\item
  政治学硕士, 美国南卡罗来纳大学\hfill 2013-03

  \begin{itemize}
  \tightlist
  \item
    \footnotesize 硕士论文: ``Connecting Nationalism and Political
    Transition: A Study of Nationalist Influence on Political Transition
    Based on the Chinese Case.''
  \end{itemize}
\item
  政治学硕士, 加拿大里贾纳大学\hfill 2011-03

  \begin{itemize}
  \tightlist
  \item
    \footnotesize 硕士论文: ``Culture and Cultural Diplomacy: A
    Comparative Study of a Canadian and Chinese Case.''
  \end{itemize}
\item
  国际关系学士, 南开大学\hfill 2009-06
\end{itemize}

\subsection{专业训练}

\begin{itemize}
\tightlist
\item
  实验研究方法 \hfill 2014-06/07

  \begin{itemize}
  \tightlist
  \item
    \footnotesize IPSA-NUS Summer School of Social Science Research
    Methods, Singapore.
  \end{itemize}
\end{itemize}

\section{工作职务}

\begin{itemize}
\tightlist
\item
  美国爱荷华大学社会研究中心统计学顾问
\end{itemize}

\section{研究兴趣}

\begin{itemize}
\tightlist
\item
  比较政治学:

  \begin{itemize}
  \tightlist
  \item
    语言政治,中国政治,公众舆论,政治传播,政治文化。
  \end{itemize}
\item
  方法论:

  \begin{itemize}
  \tightlist
  \item
    调查实验,文本分析,网络分析,空间分析,多层分析。
  \end{itemize}
\item
  国际政治:

  \begin{itemize}
  \tightlist
  \item
    公共外交,国际文化关系,中国外交政策。
  \end{itemize}
\end{itemize}

\section{发表}

\subsection{学术期刊}

Frederick Solt, Yue Hu, Kevan Hudson, Jungmin Song, et al. ``Economic
Inequality and Class Consciousness''. \emph{Journal of Politics} 3
(2017).

Frederick Solt, Yue Hu, Kevan Hudson, Jungmin Song, et al. ``Economic
Inequality and Belief in Meritocracy in the United States''.
\emph{Research and Politics} 3.4 (2016), pp.~1-7.
\url{http://rap.sagepub.com/content/3/4/2053168016672101}.

Wenfang Tang, Yue Hu and Shuai Jin. ``Affirmative Inaction: Language
Education and Labor Mobility among China's Muslim Minorities''.
\emph{Chinese Sociological Review} 4.48 (2016), pp.~346-66.

Yue Hu. ``Institutional Difference and Cultural Difference: A
Comparative Study of Canadian and Chinese Cultural Diplomacy''.
\emph{Journal of American-East Asian Relations} 20.2-3 (2013),
pp.~256-68.
\url{http://booksandjournals.brillonline.com/content/journals/10.1163/18765610-02003011}.

Yue Hu and Yuchao Zhu. ```Issue-Oriented' vs. `Ism-Oriented':
Indigenizing Political Science in China?'' {[}Chinese{]}
\emph{Tianjin Social Science} 3 (2011), pp.~46-50.

Tongshun Cheng and Yue Hu. ``The Cultural Props of Foreign Policy''.
{[}Chinese{]} \emph{Tribune of Study} (2010), pp.~33--37.

Yue Hu. ``An Analysis of Politics Fixed Position of CPPCC from the
Political Scientific Angle''. {[}Chinese{]}
\emph{Journal of Tianjin Institute of Socialism} 1 (2009), pp.~19-23.

\subsection{软件}

Frederick Solt and Yue Hu. ``pewdata: Reproducible Retrieval of Pew
Research Center Datasets''. Available at The Comprehensive R Archive
Network (CRAN). 2016. \url{http://CRAN.R-project.org/package=pewdata}.

Frederick Solt and Yue Hu. ``dotwhisker: Dot-and-Whisker Plots of
Regression Results''. Available at The Comprehensive R Archive Network
(CRAN). 2015. \url{http://CRAN.R-project.org/package=dotwhisker}.

Frederick Solt and Yue Hu. ``interplot: Plot the Effects of Variables in
Interaction Terms''. Available at The Comprehensive R Archive Network
(CRAN). 2015. \url{http://CRAN.R-project.org/package=interplot}.

\subsection{期刊在审}

Vicki Claypool, William Reisinger, Marina Zaloznaya and Yue Hu. ``The
Pernicious Effect of Petty Corruption on System-Supporting Political
Behavior in Post-Soviet States''.

Yue Hu. ``Are Informal Education Facilities Effective Means for
Political Propaganda? A Spatial Analysis''.

Yue Hu. ``Strategic Communication: Why the Chinese Government Engages in
Discourse about Democracy and Why It Matters''.

\subsection{工作论文}

``Measure Political Desirability: An Experimental Method'' (with Wenfang
Tang).

``The Weight of History: Explaining the Anti-Japanese Sentiments in the
Chinese Circle'' (with Amy Liu).

``The Logic of Peaceful Rise: Revisiting the Power-Peace Relationship in
International Politics'' (with Ray Ou-Yang)

``The Popularity of Political Propaganda in Modern China: A Model of
Demands'' (with Zijie Shao).

``Niche Audience: Chinese Government's Secret Weapon in Official Media
Propaganda'' (with Zijie Shao).

\section{会议发表}

\begin{itemize}
\tightlist
\item
  2017

  \begin{itemize}
  \tightlist
  \item
    ``The Weight of History: Explaining the Anti-Japanese Sentiments in
    the Chinese Circle'' (with Amy Liu).

    \begin{itemize}
    \tightlist
    \item
      \footnotesize 美国中西部政治学会 (MPSA), 芝加哥,伊利诺伊,
      4月7--9日。
    \end{itemize}
  \item
    ``The Popularity of Political Propaganda in Modern China: A Model of
    Demands'' (with Zijie Shao).

    \begin{itemize}
    \tightlist
    \item
      \footnotesize 美国中西部政治学会 (MPSA), 芝加哥,伊利诺伊,
      4月7--9日。
    \end{itemize}
  \item
    ``The Logic of Peaceful Rise'' (with Ray Ou-Yang).

    \begin{itemize}
    \tightlist
    \item
      \footnotesize 美国中西部政治学会 (MPSA), 芝加哥,伊利诺伊,
      4月7--9日。
    \item
      \footnotesize 国际研究年会 (ISA), 巴尔的摩,马里兰, 二月22--25日。
    \end{itemize}
  \end{itemize}
\item
  2016

  \begin{itemize}
  \tightlist
  \item
    ``Measure Political Desirability: A List Experimental Method'' (with
    Wenfang Tang);
  \item
    ``Propaganda with Museums: A Spatial Analysis of Patriotic
    Educational Demonstration Bases in China.''

    \begin{itemize}
    \tightlist
    \item
      \footnotesize 中国政治研究年会 (ACPS), 蒙特利, 加利福尼亚,
      10月10--11日。
    \end{itemize}
  \item
    ``Value Promotion, Justification, or Mobilization? The Dynamic of
    the Discourse on Democracy in Modern China.''

    \begin{itemize}
    \tightlist
    \item
      \footnotesize 美国中西部政治学会 (MPSA), 芝加哥,伊利诺伊,
      4月7-10日。
    \end{itemize}
  \end{itemize}
\item
  2015

  \begin{itemize}
  \tightlist
  \item
    ``Affirmative Inaction: Education, Social Mobility, and Ethnic
    Inequality in China'' (with Wenfang Tang and Shuai Jin).
  \item
    ``Language and Political Trust.''

    \begin{itemize}
    \tightlist
    \item
      \footnotesize 美国中西部政治学会 (MPSA), 芝加哥,伊利诺伊,
      4月16--19日。
    \end{itemize}
  \end{itemize}
\item
  2013

  \begin{itemize}
  \tightlist
  \item
    ``Connecting Nationalism and Political Transition:A Study of
    Nationalist Influence on Political Transition based on the China
    Case.''

    \begin{itemize}
    \tightlist
    \item
      \footnotesize 南开罗莱纳达研究生学会年会, 哥伦比亚,南卡罗来纳。
    \end{itemize}
  \end{itemize}
\item
  2012

  \begin{itemize}
  \tightlist
  \item
    ``Twofold-Affecting Model: A Cultural Pattern of Social Change from
    a Comparative Study between the China and Russia cases.''

    \begin{itemize}
    \tightlist
    \item
      \footnotesize 美国中国研究年会 (AACS), 奥特兰, 佐治亚,
      10月12-14日。
    \end{itemize}
  \end{itemize}
\end{itemize}

\section{研究经验}

\subsection{数据收集}

\begin{itemize}
\tightlist
\item
  网络调查实验\hfill 2017-02

  \begin{itemize}
  \tightlist
  \item
    \footnotesize 问卷、实验设计及发布,944 有效数据。
  \end{itemize}
\item
  田野调查实验 \hfill 2016-12

  \begin{itemize}
  \tightlist
  \item
    \footnotesize 双盲match-guise实验设计及实施,421有效数据.
  \end{itemize}
\item
  大数据级文本收集\hfill 2016

  \begin{itemize}
  \tightlist
  \item
    \footnotesize 编程收集《人民日报》1946至2003年发表的所有文章, 文章数
    1,371,607篇。
  \end{itemize}
\end{itemize}

\subsection{助理研究员经历}

\begin{itemize}
\tightlist
\item
  俄罗斯、乌克兰、格鲁吉亚三国腐败与政治研究
  (爱荷华大学)\hfill 2015-2016
\item
  收入差异标准化数据库项目 (爱荷华大学)\hfill 2015-05/06
\item
  东亚国家公众调查项目 (南卡罗来纳大学)\hfill 2013-05/08
\end{itemize}

\section{教学兴趣}

\begin{itemize}
\tightlist
\item
  比较政治:

  \begin{itemize}
  \tightlist
  \item
    比较政治学入门, 语言政策, 公共舆论, 亚洲政治, 中国政治, 民族学,
    民族主义, 民主化议题
  \end{itemize}
\item
  政治学方法论:

  \begin{itemize}
  \tightlist
  \item
    研究设计, 社会科学定量研究入门, 回归分析, 最大似然可能性分析,
    多层分析, 空间分析, 网络分析, 社会科学实验, 结构方程模型,
    舆论调查设计与分析, R/Stata/Python编程, 数据可视化
  \end{itemize}
\end{itemize}

\section{教学经历}

\subsection{教师,爱荷华大学}

\begin{itemize}
\tightlist
\item
  R入门系列\hfill 2015-现今
\item
  Stata与R多层分析\hfill 2016-现今
\item
  研究生新生数学预科 (合作:金帅)\hfill 2016-08
\end{itemize}

\subsection{客座讲师, 南卡罗来纳大学}\label{-}

\begin{itemize}
\tightlist
\item
  比较政治学:发展中国家\hfill 2012年春
\item
  美国国家安全\hfill 2011年秋
\end{itemize}

\subsection{教学助理, 爱荷华大学}\label{-}

\begin{itemize}
\tightlist
\item
  政治行为入门\hfill 2015年春
\item
  国际关系入门\hfill 2014年秋
\item
  比较政治学入门\hfill 2014年春
\item
  政治游说\hfill 2014年秋
\item
  政治心理学\hfill 2014年秋
\end{itemize}

\subsection{教学助理, 南卡罗来纳大学}\label{-}

\begin{itemize}
\tightlist
\item
  国际关系学\hfill 2013年春
\item
  比较政治学\hfill 2013年春
\item
  争论中的世界政治\hfill 2012年秋
\item
  美国政府\hfill 2012年春
\item
  比较政治:发展中国家\hfill 2012年春
\item
  美国国家安全\hfill 2011年秋
\end{itemize}

\subsection{教学助理, 里贾纳大学}\label{-}

\begin{itemize}
\tightlist
\item
  政治学入门\hfill 2010年秋
\end{itemize}

\section{荣誉奖励}

\subsection{研究奖励}

\begin{itemize}
\tightlist
\item
  爱荷华大学Post-Comprehensive研究奖 (\$9404.50)\hfill 2016
\item
  T. Anne Cleary国际博士论文奖学金 (\$5000) \hfill 2016
\item
  亚太研究中心博士论文研究奖金 (\$1000)\hfill 2016
\item
  爱荷华大学政治系最佳会议论文奖 (\$500)\hfill 2016
\item
  爱荷华大学政治系夏季培训补助金 (\$1900)\hfill 2014
\item
  里贾纳大学研究生科研奖金\hfill 2010
\item
  里贾纳大学研究生奖学金\hfill 2009-2011
\item
  天津市政府奖学金\hfill 2008
\end{itemize}

\subsection{学术旅行资金}

\begin{itemize}
\tightlist
\item
  研究生院理事会学术旅行奖金 (\$600)\hfill 2016
\item
  研究生政府学术旅行补助金 (\$500)\hfill 2016
\item
  亚太研究中心学术旅行补助金 (\$500)\hfill 2015
\item
  研究生院理事会学术旅行奖金 (\$200)\hfill 2015
\item
  爱荷华大学政治系学术旅行奖金
  (\$500\textasciitilde{}\$800)\hfill 2015-2017
\end{itemize}

\subsection{研究生助学金}

\begin{itemize}
\tightlist
\item
  爱荷华大学研究生助学金\hfill 2014-2017
\item
  南卡罗来纳大学研究生助学金\hfill 2011-2013
\end{itemize}

\section{组织经历}

\begin{itemize}
\tightlist
\item
  美国政治学会会员
\item
  美国中西部政治学会会员
\item
  中国政治学会会员
\item
  美国中国研究学会会员
\end{itemize}

\section{技能}

\begin{itemize}
\tightlist
\item
  分析与编程: R, Stata, Python, C++, Mathematica, NetLogo, JAG.
\item
  应用: \LaTeX, Markdown, Git(GitHub).
\end{itemize}

\section{推荐人}

\begin{tabular}{ll}
\parbox{5cm}{\textbf{Wenfang Tang }\\Professor\\University of Iowa\\(319)335-2546\\\href{mailto:wenfang-tang@uiowa.edu}{wenfang-tang@uiowa.edu}}
 &\parbox{5cm}{\textbf{William M. Reisinger}\\Professor\\University of Iowa\\(319)335-2351\\\href{mailto:william-reisinger@uiowa.edu}{william-reisinger@uiowa.edu}}\\
&\\
\parbox{5cm}{\textbf{Frederick Solt}\\Associate Professor\\University of Iowa\\(319) 335-2340\\\href{mailto:frederick-solt@uiowa.edu}{frederick-solt@uiowa.edu}}
 &\parbox{5cm}{\textbf{Caroline J. Tolbert}\\Professor\\University of Iowa\\(319) 335-2471\\\href{caroline-tolbert@uiowa.edu}{caroline-tolbert@uiowa.edu}}\\
\end{tabular}

\end{document}
