%!TEX TS-program = xelatex
%!TEX encoding = UTF-8 Unicode
% Awesome CV LaTeX Template for CV/Resume
%
% This template has been downloaded from:
% https://github.com/posquit0/Awesome-CV
%
% Author:
% Claud D. Park <posquit0.bj@gmail.com>
% http://www.posquit0.com
%
%
% Adapted to be an Rmarkdown template by Mitchell O'Hara-Wild
% 23 November 2018
%
% Template license:
% CC BY-SA 4.0 (https://creativecommons.org/licenses/by-sa/4.0/)
%
%-------------------------------------------------------------------------------
% CONFIGURATIONS
%-------------------------------------------------------------------------------
% A4 paper size by default, use 'letterpaper' for US letter
\documentclass[11pt, a4paper]{awesome-cv}

% Configure page margins with geometry
\geometry{left=1.4cm, top=.8cm, right=1.4cm, bottom=1.8cm, footskip=.5cm}

% Specify the location of the included fonts
\fontdir[fonts/]

% Color for highlights
% Awesome Colors: awesome-emerald, awesome-skyblue, awesome-red, awesome-pink, awesome-orange
%                 awesome-nephritis, awesome-concrete, awesome-darknight

\definecolor{awesome}{HTML}{414141}

% Colors for text
% Uncomment if you would like to specify your own color
% \definecolor{darktext}{HTML}{414141}
% \definecolor{text}{HTML}{333333}
% \definecolor{graytext}{HTML}{5D5D5D}
% \definecolor{lighttext}{HTML}{999999}

% Set false if you don't want to highlight section with awesome color
\setbool{acvSectionColorHighlight}{true}

% If you would like to change the social information separator from a pipe (|) to something else
\renewcommand{\acvHeaderSocialSep}{\quad\textbar\quad}

\def\endfirstpage{\newpage}

%-------------------------------------------------------------------------------
%	PERSONAL INFORMATION
%	Comment any of the lines below if they are not required
%-------------------------------------------------------------------------------
% Available options: circle|rectangle,edge/noedge,left/right

\name{胡}{悦}

\position{准聘副教授}
\address{清华大学 政治学系}

\email{\href{mailto:yuehu@tsinghua.edu.cn}{\nolinkurl{yuehu@tsinghua.edu.cn}}}
\homepage{sammo3182.github.io}
\github{sammo3182}
\twitter{yuehupolisci}

% \gitlab{gitlab-id}
% \stackoverflow{SO-id}{SO-name}
% \skype{skype-id}
% \reddit{reddit-id}


\usepackage{booktabs}

\providecommand{\tightlist}{%
	\setlength{\itemsep}{0pt}\setlength{\parskip}{0pt}}

%------------------------------------------------------------------------------


\usepackage{ctex}

% Pandoc CSL macros
\newlength{\cslhangindent}
\setlength{\cslhangindent}{1.5em}
\newlength{\csllabelwidth}
\setlength{\csllabelwidth}{3em}
\newenvironment{CSLReferences}[3] % #1 hanging-ident, #2 entry spacing
 {% don't indent paragraphs
  \setlength{\parindent}{0pt}
  % turn on hanging indent if param 1 is 1
  \ifodd #1 \everypar{\setlength{\hangindent}{\cslhangindent}}\ignorespaces\fi
  % set entry spacing
  \ifnum #2 > 0
  \setlength{\parskip}{#2\baselineskip}
  \fi
 }%
 {}
\usepackage{calc}
\newcommand{\CSLBlock}[1]{#1\hfill\break}
\newcommand{\CSLLeftMargin}[1]{\parbox[t]{\csllabelwidth}{#1}}
\newcommand{\CSLRightInline}[1]{\parbox[t]{\linewidth - \csllabelwidth}{#1}}
\newcommand{\CSLIndent}[1]{\hspace{\cslhangindent}#1}

\begin{document}

% Print the header with above personal informations
% Give optional argument to change alignment(C: center, L: left, R: right)
\makecvheader

% Print the footer with 3 arguments(<left>, <center>, <right>)
% Leave any of these blank if they are not needed
% 2019-02-14 Chris Umphlett - add flexibility to the document name in footer, rather than have it be static Curriculum Vitae
\makecvfooter
  {July 2021}
    {胡 悦~~~·~~~Curriculum Vitae}
  {\thepage}


%-------------------------------------------------------------------------------
%	CV/RESUME CONTENT
%	Each section is imported separately, open each file in turn to modify content
%------------------------------------------------------------------------------



\hypertarget{ux6559ux80b2ux7ecfux5386}{%
\section{教育经历}\label{ux6559ux80b2ux7ecfux5386}}

\begin{cventries}
    \cventry{爱荷华大学(University of Iowa, Iowa City)}{哲学博士(Doctor of Philosophy, Political Science)}{美国,爱荷华}{2013-2018}{}\vspace{-4.0mm}
    \cventry{南卡罗莱纳大学(University of South Carolina, Columbia)}{文学硕士(Master of Arts,Political Science)}{美国,南卡罗莱纳}{2011-2013}{}\vspace{-4.0mm}
    \cventry{里贾纳大学(University of Regina, Regina)}{文学硕士(Master of Arts,Political Science)}{加拿大,里贾纳}{2009-2011}{}\vspace{-4.0mm}
    \cventry{南开大学}{法学学士(国际关系)}{中国,天津}{2005-2009}{}\vspace{-4.0mm}
\end{cventries}

\hypertarget{ux4e13ux4e1aux8badux7ec3}{%
\subsection{专业训练}\label{ux4e13ux4e1aux8badux7ec3}}

\begin{cventries}
    \cventry{爱荷华大学(University of Iowa, Iowa City)}{信息学辅修证书(Certificate of Informatics)}{美国,爱荷华州}{2016-12}{}\vspace{-4.0mm}
    \cventry{新加坡国立大学}{实验研究方法(Experimental Method)}{新加坡}{2014-06}{}\vspace{-4.0mm}
\end{cventries}

\hypertarget{ux5de5ux4f5cux7ecfux5386}{%
\section{工作经历}\label{ux5de5ux4f5cux7ecfux5386}}

\begin{cventries}
    \cventry{清华大学计算社会科学平台}{副主任}{中国,北京}{2020-07至今}{}\vspace{-4.0mm}
    \cventry{Github}{校园导师(Campus Advisor)}{美国,加利福尼亚州}{2021-07}{}\vspace{-4.0mm}
    \cventry{清华大学社会科学院政治学系}{准聘副教授}{中国,北京}{2020-06至今}{}\vspace{-4.0mm}
    \cventry{清华大学社会科学院}{党委研工组组长}{中国,北京}{2020-09至今}{}\vspace{-4.0mm}
    \cventry{清华大学数据与治理中心}{副主任}{中国,北京}{2019-06至今}{}\vspace{-4.0mm}
    \cventry{清华大学社会科学院政治学系}{助理教授}{中国,北京}{2019-01至2021-05}{}\vspace{-4.0mm}
    \cventry{爱荷华社会科学研究中心(Iowa Social Science Research Center)}{统计学顾问(Statistical Consultant)}{美国,爱荷华州}{2015-05 ~ 2018-05}{}\vspace{-4.0mm}
\end{cventries}

\hypertarget{ux8bbaux6587ux53d1ux8868}{%
\section{论文发表}\label{ux8bbaux6587ux53d1ux8868}}

\hypertarget{peer-review-ux8bbaux6587}{%
\subsection{\texorpdfstring{\textbf{Peer-Review
论文}}{Peer-Review 论文}}\label{peer-review-ux8bbaux6587}}

\begingroup
\setlength{\parindent}{-0.5in}
\setlength{\leftskip}{0.5in}

\hypertarget{refs_published}{}
\leavevmode\hypertarget{ref-HuPizzi2022}{}%
Hu, Y., \& Pizzi, E. (2022). The {Role} of {Language Policy} in
{Migration Decisions}. \emph{China: An International Journal}, Accepted.

\leavevmode\hypertarget{ref-PizziHu2021}{}%
Pizzi, E., \& Hu, Y. (2021). Does {Governmental Policy Shape Migration
Decisions}?{The Case} of {China}'s hukou {System}. \emph{Modern China},
Accepted.

\leavevmode\hypertarget{ref-Hu2020}{}%
Hu, Y. (2020). Culture {Marker Versus Authority Marker}: {How Do
Language Attitudes Affect Political Trust}? \emph{Political Psychology},
\emph{41}(4), 699--716. \url{https://doi.org/10.1111/pops.12646}

\leavevmode\hypertarget{ref-Hu2020a}{}%
Hu, Y. (2020). Refocusing {Democracy}: {The Chinese Government}'s
{Framing Strategy} in {Political Language}. \emph{Democratization},
\emph{72}(2), 302--320.
\url{https://doi.org/10.1080/13510347.2019.1690461}

\leavevmode\hypertarget{ref-HuLiu2020}{}%
Hu, Y., \& Liu, A. H. (2020). The {Effects} of {Foreign Language
Proficiency} on {Public Attitudes}: {Evidence} from the
{Chinese}-{Speaking World}. \emph{Journal of East Asian Studies},
\emph{20}(1), 1--23. \url{https://doi.org/10.1017/jea.2019.41}

\leavevmode\hypertarget{ref-Hu2019}{}%
Hu, Y. (2019). Are {Informal Education Facilities Effective Means} for
{Generating Political Support}? {A Spatial Analysis}. \emph{Social
Science Quarterly}, \emph{100}(3), 701--724.
\url{https://doi.org/10.1111/ssqu.12589}

\leavevmode\hypertarget{ref-ClaypoolEtAl2018}{}%
Claypool, V. H., Reisinger, W., Zaloznaya, M., Hu, Y., \& Juehring, J.
(2018). Tsar {Putin} and the "{Corruption}" {Thorn} in his {Side}: {The
Demobilization} of {Votes} in a {Competitive Authoritarian Regime}.
\emph{Electoral Studies}, \emph{54}, 182--204.

\leavevmode\hypertarget{ref-Hu2018b}{}%
Hu, Y. (2018). \emph{Rebuilding the {Tower} of {Babel}: {Language
Policy} and {Political Trust} in {China}} {[}PhD thesis{]}. The
University of Iowa.

\leavevmode\hypertarget{ref-SoltEtAl2017}{}%
Solt, F., Hu, Y., Hudson, K., Song, J., \& Yu, D. "Erico". (2017).
Economic {Inequality} and {Class Consciousness}. \emph{The Journal of
Politics}, \emph{79}(3), 1079--1083.
\url{https://doi.org/10.1086/690971}

\leavevmode\hypertarget{ref-SoltEtAl2016}{}%
Solt, F., Hu, Y., Hudson, K., Song, J., \& Yu, D. "Erico". (2016).
Economic {Inequality} and {Belief} in {Meritocracy} in the {United
States}. \emph{Research \& Politics}, \emph{3}(4), 1--7.
\url{https://doi.org/10.1177/2053168016672101}

\leavevmode\hypertarget{ref-TangEtAl2016a}{}%
Tang, W., Hu, Y., \& Jin, S. (2016). Affirmative {Inaction}: {Language
Education} and {Labor Mobility} among {China}'s {Muslim Minorities}.
\emph{Chinese Sociological Review}, \emph{48}(4), 346--366.

\leavevmode\hypertarget{ref-Hu2013}{}%
Hu, Y. (2013). Institutional {Difference} and {Cultural Difference}: {A
Comparative Study} of {Canadian} and {Chinese Cultural Diplomacy}.
\emph{Journal of American-East Asian Relations}, \emph{20}(2-3),
256--268. \url{https://doi.org/10.1163/18765610-02003011}

\leavevmode\hypertarget{ref-Hu2011}{}%
Hu, Y. (2011). \emph{Culture and {Cultural Diplomacy}: {A Comparative
Study} of a {Canadian} and {Chinese Case}} {[}PhD thesis{]}. The
University of Regina.

\leavevmode\hypertarget{ref-Hu2009}{}%
Hu, Y. (2009). {An Analysis of Politics Fixed Position of CPPCC from the
Political Scientific Angle}. \emph{Journal of Tianjin Institute of
Socialism}, \emph{1}, 19--23.

\endgroup

\hypertarget{ux4e2dux6587ux8bbaux6587}{%
\subsection{\texorpdfstring{\textbf{中文论文}}{中文论文}}\label{ux4e2dux6587ux8bbaux6587}}

\begingroup
\setlength{\parindent}{-0.5in}
\setlength{\leftskip}{0.5in}

\hypertarget{refs_chinese}{}
\leavevmode\hypertarget{ref-HuYueZhuYuZhao2011}{}%
胡悦, \& 朱毓朝. (2011).
{{``问题与主义''}导向之争{}{}西方比较政治学沿革及其借鉴意义}.
\emph{天津社会科学}, \emph{03}, 46--50.

\leavevmode\hypertarget{ref-ChengTongShunHuYue2010}{}%
程同顺, \& 胡悦. (2010). {对外政策的文化道具{}{}浅析文明冲突论的工具性}.
\emph{学习论坛}, \emph{26}(02), 33--37.

\leavevmode\hypertarget{ref-HuYue2009}{}%
胡悦. (2009). {从政治科学角度分析人民政协的政治定位}.
\emph{天津市社会主义学院学报}, \emph{01}, 19--23.

\endgroup

\hypertarget{ux7814ux53d1ux8f6fux4ef6}{%
\subsection{\texorpdfstring{\textbf{研发软件}}{研发软件}}\label{ux7814ux53d1ux8f6fux4ef6}}

\begingroup
\setlength{\parindent}{-0.5in}
\setlength{\leftskip}{0.5in}

\hypertarget{refs_software}{}
\leavevmode\hypertarget{ref-HuEtAl2021}{}%
Hu, Y., Sun, Y., \& Wu, W. (2021). \emph{Regioncode: {Convertor} of
{Prefectural}-{Level Administrative Region Names}-{Codes}}.

\leavevmode\hypertarget{ref-Hu2021}{}%
Hu, Y. (2021). \emph{Drhutools: {Toolbox} for {Writing} an {Academic
Paper} of {Empirical Research} with {Rmarkdown}}.

\leavevmode\hypertarget{ref-HuSun2020}{}%
Hu, Y., \& Sun, Y. (2020). \emph{Drhur: {Interactive Platform} for
{Liberal Arts Students} to {Learn R}}.

\leavevmode\hypertarget{ref-SoltHu2016}{}%
Solt, F., \& Hu, Y. (2016). \emph{Pewdata: {Reproducible Retrieval} of
{Pew Research Center Datasets}}. Available at The Comprehensive R
Archive Network (CRAN).

\leavevmode\hypertarget{ref-SoltHu2015}{}%
Solt, F., \& Hu, Y. (2015). \emph{Dotwhisker: {Dot}-and-{Whisker Plots}
of {Regression Results}}. Available at The Comprehensive R Archive
Network (CRAN).

\leavevmode\hypertarget{ref-SoltHu2015a}{}%
Solt, F., \& Hu, Y. (2015). \emph{Interplot: {Plot} the {Effects} of
{Variables} in {Interaction Terms}}. Available at The Comprehensive R
Archive Network (CRAN).

\endgroup

\hypertarget{ux79d1ux7814ux9879ux76ee}{%
\section{科研项目}\label{ux79d1ux7814ux9879ux76ee}}

\begin{cventries}
    \cventry{国家自然科学基金}{新型城镇化进程中新老市民身份认同建构的社会心理机制与政策引导路径研究}{}{2021}{}\vspace{-4.0mm}
    \cventry{北京市社会科学基金}{突发公共卫生事件互联网政治生态与话语竞争研究}{}{2020}{}\vspace{-4.0mm}
    \cventry{北京是规划和自然资源委员会}{ 北京市违法建设专题研究}{}{2020}{}\vspace{-4.0mm}
\end{cventries}

\hypertarget{ux6559ux5b66ux7ecfux5386}{%
\section{教学经历}\label{ux6559ux5b66ux7ecfux5386}}

\begin{cventries}
    \cventry{}{70700173: 治理技术专题:政治数据分析}{清华大学}{秋季学期}{}\vspace{-4.0mm}
    \cventry{}{80700673: 政务大数据应用与分析}{清华大学}{秋季学期}{}\vspace{-4.0mm}
    \cventry{}{10700193: 理解公共政策:多元视角与案例解析}{清华大学}{春季学期}{}\vspace{-4.0mm}
    \cventry{}{30700953: 公共政策分析:视角与方法}{清华大学}{春季学期}{}\vspace{-4.0mm}
    \cventry{}{定量学群R语言工作坊}{清华大学}{全年课程}{}\vspace{-4.0mm}
    \cventry{}{POLI301: 政治数据分析}{爱荷华大学}{2018 秋季学期}{}\vspace{-4.0mm}
\end{cventries}

\hypertarget{ux4f1aux8baeux6d3bux52a8}{%
\section{会议活动}\label{ux4f1aux8baeux6d3bux52a8}}

\begingroup
\setlength{\parindent}{-0.5in}
\setlength{\leftskip}{0.5in}

\hypertarget{refs_conference}{}
\leavevmode\hypertarget{ref-Hu2021a}{}%
Hu, Y. (2021, July). Linguistic {Relativity} of {Political Trust}:
{Effects} and {Mechanisms}. \emph{The 26th {International Political
Science Association}}.

\leavevmode\hypertarget{ref-HuYueZhuMeng2021}{}%
胡悦, \& 朱萌. (2021, July).
语言政策场域:外语习得对政治效能塑造机制研究.
\emph{政治学与国际关系学术共同体会议}.

\leavevmode\hypertarget{ref-HuEtAl2021a}{}%
Hu, Y., Sun, Y., \& Su, Y.-S. (2021, May). Diverse {Loyalty}:
{Psychological Paths} of {Mass Response} to {Political News}.
\emph{第四届中国政治传播研究学术论坛}.

\leavevmode\hypertarget{ref-TaiEtAl2021}{}%
Tai, Y., Hu, Y., \& Frederick, S. (2021, January). Democracy, {Public
Support}, and {Measurement Uncertainty}. \emph{{AsianPolmeth VIII} \&
{ASQPS IX}}.

\leavevmode\hypertarget{ref-JiangQiaoLeiHuYue2020}{}%
蒋俏蕾, \& 胡悦. (2020, November). 青少年饭圈身份认同与集体行动:
基于观察与实验数据的实证考察. \emph{中国心理学会}.

\leavevmode\hypertarget{ref-JinEtAl2020}{}%
Jin, S., Hu, Y., \& Meng, T. (2020, October). How {Inequality Affects
Political Trust}: {A Mediation Analysis}. \emph{{``中国政治 新锐研究''}
学术研讨会}.

\leavevmode\hypertarget{ref-HuYue2019a}{}%
胡悦. (2019, November). 新型城镇化中的语言治理:个体、群体与国家.
\emph{第六届{``政治科学前沿理论与方法''}论坛}.

\leavevmode\hypertarget{ref-PizziHu2019}{}%
Pizzi, E., \& Hu, Y. (2019, August). Are {Strict Rules} a {Deterrent}?
{Political Restrictions} and {Migration} in {China}. \emph{American
{Political Science Association}'s 2019 {Annual Meeting} and {Exhibition}
({APSA})}.

\leavevmode\hypertarget{ref-HuYue2019b}{}%
胡悦. (2019, May). 数据可视化与反p-{Hacking}:以政治科学研究为例.
\emph{{中国R会议}}.

\leavevmode\hypertarget{ref-HuYueShaoZiJie2019}{}%
胡悦, \& 邵梓捷. (2019, May). Supply vs. {Demand}: {Why People Watch
State}-{Media Broadcasts} in {China}.
\emph{政治传播与社会发展论坛暨``政治传播:
回到原点,守正创新``学术研讨会}.

\leavevmode\hypertarget{ref-TangHu2018}{}%
Tang, W., \& Hu, Y. (2018, June). Informal {Economic Behavior} and
{Authoritarian Regime Instability}. \emph{Workshop on {Chinese Politics}
and {Society}}.

\leavevmode\hypertarget{ref-Hu2018a}{}%
Hu, Y. (2018, April). Rebuilding the {Tower} of {Babel}? {The Influence}
of {Language Policy} on {Political Trust}. \emph{The 76th {Midwest
Political Science Association} ({MPSA}) {Annual Convention}}.

\leavevmode\hypertarget{ref-HuTang2016a}{}%
Hu, Y., \& Tang, W. (2016, October). Measure {Political Desirability}:
{An Experimental Method}. \emph{The 30nd {Association} of {Chinese
Political Studies} ({ACPS}) {Annual Meeting}}.

\endgroup

\hypertarget{ux8363ux8a89ux5956ux52b1}{%
\section{荣誉奖励}\label{ux8363ux8a89ux5956ux52b1}}

\begin{cventries}
    \cventry{爱荷华大学}{Ballard and Seashore 奖学金}{}{2017}{}\vspace{-4.0mm}
    \cventry{爱荷华大学}{T. Anne Cleary 国际博士论文奖学金}{}{2016}{}\vspace{-4.0mm}
    \cventry{爱荷华大学}{ Post-Comprehensive 研究奖}{}{2016}{}\vspace{-4.0mm}
\end{cventries}

\hypertarget{ux4e13ux4e1aux6280ux80fd}{%
\section{专业技能}\label{ux4e13ux4e1aux6280ux80fd}}

\hypertarget{ux5206ux6790ux4e0eux7f16ux7a0b}{%
\subsection{\texorpdfstring{\textbf{分析与编程}}{分析与编程}}\label{ux5206ux6790ux4e0eux7f16ux7a0b}}

R, STATA, Python, C++, Mathematica, NetLogo, JAGS, UCINET

\hypertarget{ux5e94ux7528}{%
\subsection{\texorpdfstring{\textbf{应用}}{应用}}\label{ux5e94ux7528}}

LaTeX, Markdown, Git

\hypertarget{ux5b66ux672fux5171ux540cux4f53}{%
\section{学术共同体}\label{ux5b66ux672fux5171ux540cux4f53}}

\begin{itemize}
\tightlist
\item
  美国政治科学协会(APSA)会员
\item
  中西部政治科学协会(MPSA)会员
\item
  中国政治科学研究协会(ACPS)会员
\item
  国际政治科学协会(IPSA)会员
\end{itemize}

\end{document}
